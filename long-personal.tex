\documentclass[12pt]{article}

\usepackage{setspace}
\usepackage[margin=1in]{geometry}

\singlespacing

\setlength{\parskip}{6pt}

\begin{document}
\noindent My name is Connor Briggs, and I am a graduating senior studying Chemistry and Mathematics.  I have always been interested in chemistry, mathematics, and computer science.  Quantum chemistry is at the confluence of these areas, and I hope be able to contribute to future work, especially in the areas of machine learning applied to the computational methods at the heart of the discipline.\par

Since my freshman year, I have worked under Dr. T. Daniel Crawford at Virginia Tech researching applications, developing software for the MolSSI, and applying strong computing principles including design patterns and algorithm implementation to support others in our team.  Our work has been presented nationally and internationally, and even includes the prestigious Journal of Physical Chemistry where I was named as second author on a paper about using machine learning for quantum properties.\par

I am interested in continuing quantum chemistry research, especially in developing new methods for computing properties.  One of the major challenges in this area is computational intractability of direct solutions, so I am very interested in exploring new AI techniques to reduce computation time while bounding error rates. My past research in this problem has shown promising results, and current research in the possibility of training models on large data sets provided by the QCArchive has given encouraging results.\par

My research in quantum chemistry has guided me to look into the analysis of the methods used. It has encouraged me to take several extra courses in mathematics which can be applied to analyze algorithms. I have also learned new programming languages which have helped me in processing the data obtained from calculations and writing new code for calculations. This has, in turn, guided my interests in my research. This back-and-forth has made me a more well-rounded researcher and student.\par

I am a strong programmer, and have been writing code for over a decade. Over that time, I have learned to program fluently in C/C++, Python, Java, and FORTRAN, and I am able to quickly pick up new languages if needed. More recently, I have been contributing code to the Psi4 repository on GitHub relating to the possibility of data-mining quantum chemical properties. However, I am not so strong on the theory side. This is mostly due to lack of knowledge, as I have never had a class in this field. My experience has been solving programming problems for Dr. Crawford, which included writing Hartree-Fock and CCSD methods in C++. I would hope that over the course of my degree, I will become more comfortable with the theories used in this field.\par

My first semesters at Virginia Tech were an extremely difficult adjustment for me.  Living on my own brought new challenges, and because of this I failed two of my courses.  This low point was the start of a positive trajectory for me, personally and academically.  I learned how to persevere, work hard, and most importantly, to reach out for help when needed.  Despite this positive personal growth, these early failures have been a constant drag on my GPA. As a predictor of my readiness for graduate study, I feel that my rich research experience coupled with my most recent semesters, and GRE scores paint a more accurate picture.


\end{document}
