\documentclass[12pt]{article}

\usepackage{setspace}
\usepackage[margin=1in]{geometry}

\doublespacing

\begin{document}
\noindent My name is Connor Briggs, and I am a graduating senior studying Chemistry and Mathematics.  I have always been interested in chemistry, mathematics, and computer science.  Quantum chemistry is at the confluence of these areas, and I hope be able to contribute to future work, especially in the areas of machine learning applied to the computational methods at the heart of the discipline.

Since my freshman year, I worked under Dr. T. Daniel Crawford at Virginia Tech researching applications, developing software for the MolSSI, and applying strong computing principles including design patterns and algorithm implementation to support others in our team.  Our work has been presented nationally and internationally, and even includes the prestigious Journal of Physical Chemistry where I was named as second author.

I am interested in continuing quantum chemistry research, especially in developing new methods for computing properties.  One of the major challenges in this area is computational intractability of direct solutions, so I am very interested in exploring new AI techniques to reduce computation time while bounding error rates.

The University of Pennsylvania was recommended to me by several of my peers.  In particular, I am looking to study under \textbf{Dr. Joseph Subotnik}, \textbf{Dr. Andrew Rappe}, or \textbf{Dr. Jeffrey Saven}. They are active in the area of physical chemistry and methods analysis, and I would welcome the opportunity to support their work and hope to extend it through the AI techniques I have learned under Dr. Crawford.

My first semesters at Virginia Tech were an extremely difficult adjustment for me.  Living on my own brought new challenges, and because of this I failed two of my courses.  This low point was the start of a positive trajectory for me, personally and academically.  I learned how to persevere, work hard, and most importantly, to reach out for help when needed.  Despite this positive personal growth, these early failures have been a constant drag on my GPA. As a predictor of my readiness for graduate study, I feel that my rich research experience coupled with my most recent semesters, and GRE scores paint a more accurate picture.


\end{document}
